\documentclass{book}[oneside]
\usepackage{CJKutf8}
\usepackage{amsmath}
\usepackage{amsfonts}
\usepackage{amsthm}
\usepackage{titlesec}
\usepackage{titletoc}
\usepackage{xCJKnumb}
\usepackage{tikz}
\titleformat{\chapter}{\centering\Huge\bfseries}{第\, \xCJKnumber{\thechapter}\,
    章}{1em}{}
  % \renewcommand{\chaptermark}[1]{\markboth{第 \thechapter 章}{}}
\usepackage{mathrsfs}

\newtheorem{Def}{定义}[chapter]
\newtheorem{Thm}{定理}[chapter]
\newtheorem{Exercise}{练习}[chapter]

\newtheorem*{Example}{例}


\begin{document}
\begin{CJK*}{UTF8}{gbsn}
  \title{离散数学讲义}
  \author{陈建文}
  \maketitle
  % \tableofcontents
  
  \underline{课程学习目标:}
\begin{enumerate}
\item 训练自己的逻辑思维能力和抽象思维能力
\item 训练自己利用数学语言准确描述计算机科学问题和电子信息科学问题的能力
\end{enumerate}

\underline{学习方法:}
\begin{enumerate}
\item MOOC自学
\item 阅读该讲义
\item 做习题
\item 学习过程中有不懂的问题,在课程QQ群中与老师交流
\end{enumerate}

\underline{授课教师QQ:}2129002650


  \setcounter{chapter}{1}
  \chapter{映射}

  \begin{Def}
    设$X$和$Y$为两个非空集合。一个从$X$到$Y$的映射$f$为一个法则,根据$f$,对$X$中的每个元素$x$都有$Y$中唯一确定的元素$y$与之对应。
    从$X$到$Y$的映射$f$常记为$f:X\to Y$。
  \end{Def}

  \begin{Example}
    设集合$X=\{-1,0,1\}$,集合$Y=\{0,1,2\}$,$\forall x \in X, f(x)=x^2$,即$f(-1)=1,f(0)=0,f(1)=1$,则$f$为从集合$X$到集合$Y$的映射。
  \end{Example}

  \begin{Def}
    设$X$和$Y$为两个非空集合。一个从$X$到$Y$的映射为一个满足以下两个条件的$X\times Y$的子集$f$:
    \begin{enumerate}
    \item 对$X$的每一个元素$x$,存在一个$y\in Y$,使得$(x,y) \in f$;
    \item 若$(x,y)\in f$,$(x,y')\in f$,则$y=y'$。
    \end{enumerate}
    $(x,y)\in f$记为$y=f(x)$。
  \end{Def}
  \begin{Example}
    设集合$X=\{-1,0,1\}$,集合$Y=\{0,1,2\}$,$f=\{(-1,1),(0,0),(1,1)\}$,则$f$为从集合$X$到集合$Y$的映射。
  \end{Example}

  定义2.1和定义2.2是等价的。
  \begin{Def}
    设$f$为从集合$X$到集合$Y$的映射,$f:X\to Y$, 如果$y = f(x)$,则称$y$为$x$在$f$下的象,称$x$为$y$的原象。$X$称为$f$的定义域;集合$\{f(x) | x \in X\}$称为$f$的值域,记为$Im(f)$。
  \end{Def}
  \begin{Def}
    设$f:X\to Y$,$A\subseteq X$,当把$f$的定义域限制在$A$上时,就得到了一个
    $\phi: A\to Y$,$\forall x \in A$,$\phi(x) = f(x)$。$\phi$称为$f$在$A$上的
    限制,并且常用$f|A$来表示$\phi$。反过来,我们也称$f$为$\phi$在$X$上的扩张。
  \end{Def}
    \begin{Def}
    设$f:A \to Y$,$A \subseteq X$, 则称$f$为$X$上的一个部分映射。
  \end{Def}
  \begin{Def}
    两个映射$f$与$g$称为是相等的当且仅当$f$和$g$都为$X$到$Y$的映射,并且$\forall x \in X$总有$f(x) = g(x)$。
  \end{Def}
  \begin{Def}
    设$f:X\to X$,如果$\forall x \in X, f(x) = x$,则称$f$为$X$上的恒等映射。$X$上的恒等映射常记为$I_X$。
  \end{Def}

    \begin{Def}
    设$f:X\to Y$,如果$\forall x_1, x_2 \in X$, 只要$x_1 \neq x_2$,  就 有 $f(x_1) \neq f(x_2)$,   则称$f$为从$X$到$Y$的单射。
  \end{Def}
  \begin{Def}
    设$f:X\to Y$, 如果$\forall y \in Y$, $\exists x \in X$使得 $f(x) = y$, 则称$f$为从$X$到$Y$的满射。
  \end{Def}
  \begin{Def}
    设$f:X\to Y$,如果$f$既是单射又是满射,则称$f$为从$X$到$Y$的双射,或者称$f$为从$X$到$Y$的一一对应。
  \end{Def}
  \begin{Def}
    设$f:X\to Y$,$A \subseteq X$,$A$在$f$下的象定义为\[f(A)=\{f(x)|x\in A\}\]
  \end{Def}
  \begin{Example}
    设$f:\{-1,0,1\}\to \{-1,0,1\}$,$f(x)=x^2$,则$f(\{-1,0\})=\{0,1\}$
  \end{Example}
  \begin{Def}
    设$f:X\to Y$,$B \subseteq Y$,$B$在$f$下的原象定义为\[f^{-1}(B)=\{x\in X|f(x)\in B\}\]
  \end{Def}
  \begin{Example}
    设$f:\{-1,0,1\}\to \{-1,0,1\}$,$f(x)=x^2$,则$f^{-1}(\{-1,0\})=\{0\}$
  \end{Example}
  \begin{Thm}
    设$f:X\to Y$,$A \subseteq Y$,$B \subseteq Y$, 则
    \begin{enumerate}
    \item $f^{-1}(A \cup B) = f^{-1}(A) \cup f^{-1}(B)$
    \item $f^{-1}(A \cap B) = f^{-1}(A) \cap f^{-1}(B)$
    \item $f^{-1}(A^c)=(f^{-1}(A))^c$
    \item $f^{-1}(A \bigtriangleup B) = f^{-1}(A) \bigtriangleup f^{-1}(B)$
    \end{enumerate}
  \end{Thm}
    \begin{Thm}
    设$f:X\to Y$,$A \subseteq X$,$B \subseteq X$, 则
    \begin{enumerate}
    \item $f(A \cup B) = f(A) \cup f(B)$
    \item $f(A \cap B) \subseteq f(A) \cap f(B)$
    \item $f(A \bigtriangleup B) \supseteq f(A) \bigtriangleup f(B)$
    \end{enumerate}
  \end{Thm}

  \begin{Def}
    设$f:X\to Y$,$g:Y\to Z$为映射,映射$f$与$g$的合成$g\circ f:X\to Z$定义为\[(g\circ f)(x) = g(f(x))\]
  \end{Def}
\begin{Thm}
  设$f:X \to Y$,$g:Y\to Z$,$h:Z\to W$ 为映射,则 \[ (h \circ g) \circ f = h \circ (g \circ f) \]
\end{Thm}

  \begin{Def}
     设$f:X\to Y$为双射,$f$的逆映射$f^{-1}:Y\to X$定义为:对任意的$y\in Y$,存在唯一的$x$使得$f(x)=y$,则$f^{-1}(y)=x$。
   \end{Def}
    \begin{Def}
     设$f:X\to Y$为一个映射。如果存在一个映射$g:Y\to X$使得\[f\circ g = I_{Y} \text{且} g\circ f = I_{X},\]则称映射$f$为可逆的,而$g$称为$f$的逆映射。
   \end{Def}
   以上两个定义是等价的。

     \begin{Thm}
    设$f:X\to Y$为可逆映射,则$(f^{-1})^{-1}=f$。
  \end{Thm}
  \begin{Thm}
    设$f:X\to Y$,$g:Y\to Z$都为可逆映射,则$g\circ f$也为可逆映射并且$(g\circ f)^{-1} = f^{-1}\circ g^{-1}$。
  \end{Thm}

    \begin{Def}
    设$f:X\to Y$为一个映射,如果存在一个映射 $g:Y\to X$ 使 得 $g\circ f = I_X$,
    则称$f$为左可逆的,$g$称为$f$的左逆映射;如果存在一个映射
    $h:Y\to X$ 使 得 $f\circ h=I_Y$,则称$f$为右可逆的,$h$称为$f$的右逆映射。
  \end{Def}
  \begin{Thm}
    设$f:X\to Y$为一个映射,则
    \begin{enumerate}
    \item $f$左可逆当且仅当$f$为单射;
    \item $f$右可逆当且仅当$f$为满射。
    \end{enumerate}
  \end{Thm}

  \begin{Def}
    有穷集合$S$到自身的一一对应称为$S$上的一个置换。如果$|S| = n$, 则$S$上的置换就说成是$n$次置换。
  \end{Def}
设$S=\{1,2,\ldots,n\}$,$\sigma:S\to S$为$S$上的一个置换,$\sigma(1) = k_1$, $\sigma(2) = k_2$,$\ldots$,$\sigma(n) = k_n$,我们用如下的一个表来表示置换$\sigma$:
\[\sigma=\begin{pmatrix}1&2&\ldots&n\\k_1&k_2&\ldots&k_n\end{pmatrix}\]
\begin{Example}
  设$S=\{1,2,3,4\}$,$\sigma(1) = 3$, $\sigma(2) = 2$, $\sigma(3) = 4$, $\sigma(4) = 1$,则$\sigma$可以表示为
  \[\sigma=\begin{pmatrix}1&2&3&4\\3&2&4&1\end{pmatrix}\]
  这里,列的次序无关紧要,例如,$\sigma$还可以表示为
  \[\sigma=\begin{pmatrix}2&1&3&4\\2&3&4&1\end{pmatrix}\]
\end{Example}

  \begin{Def}
    设$\alpha$与$\beta$为集合$S=\{1,2,3,4\}$上的两个置换,则$\alpha$与$\beta$为两个从$S$到$S$的双射,讨论置换时,我们用$\alpha\beta$表示$\alpha$与$\beta$的合成$\beta \circ \alpha$。
    注意这里$\alpha$与$\beta$的次序,从运算的角度看有一定的便利性,但也有的教材中采用相反的顺序。按照我们的写法,讨论置换时,如果$i \in S$,则用$(i)\alpha$表示$i$在$\alpha$下的像,简记为$i\alpha$。
  \end{Def}
  \begin{Example}
    设$S=\{1,2,3\}$,$\alpha$和$\beta$为$S$上的两个置换,
    \[\alpha=\begin{pmatrix}1&2&3\\1&3&2\end{pmatrix},\beta=\begin{pmatrix}1&2&3\\2&3&1\end{pmatrix}\],
    则
    \[\alpha\beta=\begin{pmatrix}1&2&3\\1&3&2\end{pmatrix}\begin{pmatrix}1&2&3\\2&3&1\end{pmatrix}=\begin{pmatrix}1&2&3\\2&1&3\end{pmatrix}\],    
  \end{Example}

   若$\alpha$与$\beta$为两个$n$次置换,当把$\beta$的表示式中的上一行按$\alpha$的下一行的顺序写出时,则$\alpha \beta$的下一行就是$\beta$的新表示式中的下一行。
  \begin{Example}
    设$S=\{1,2,3\}$,$\alpha$和$\beta$为$S$上的两个置换,
    \[\alpha=\begin{pmatrix}1&2&3\\1&3&2\end{pmatrix},\beta=\begin{pmatrix}1&2&3\\2&3&1\end{pmatrix}\],
    则
    \[\alpha\beta=\begin{pmatrix}1&2&3\\1&3&2\end{pmatrix}\begin{pmatrix}1&2&3\\2&3&1\end{pmatrix}=\begin{pmatrix}1&2&3\\1&3&2\end{pmatrix}\begin{pmatrix}1&3&2\\2&1&3\end{pmatrix}=\begin{pmatrix}1&2&3\\2&1&3\end{pmatrix}\],    
  \end{Example}   
   \begin{Def}
     设$\sigma$为$S$上的一个$n$次置换,若$i_1\sigma=i_2$,$i_2\sigma = i_3$, $\cdots$, $i_{k-1}\sigma = i_k$, $i_k\sigma = i_1$,而$\forall i \in S\setminus \{i_1, i_2, \ldots, i_k\}$, $i\sigma = i$,
     则称$\sigma$为一个$k$循环置换,记为$(i_1i_2\cdots i_k)$。 $2-$循环置换称为对换。
   \end{Def}
   \begin{Example}
   设$S=\{1,2,3,4,5\}$,则\[(1,2,3)=\begin{pmatrix}1&2&3&4&5\\2&3&1&4&5\end{pmatrix},(2,3)=\begin{pmatrix}1&2&3&4&5\\1&3&2&4&5\end{pmatrix}\]     
   \end{Example}
   \begin{Thm}
    每个置换都能被分解成若干个没有共同数字的循环置换的乘积。如果不计这些循环置换的顺序以及略去的$1-$循环置换,这个分解是唯一的。
   \end{Thm}
   \begin{Thm}
    当$n\geq 2$时,每个$n$次置换都能被分解成若干个对换的乘积。
   \end{Thm}
   \begin{Thm}
    如果把置换分解成若干个对换的乘积,则对换个数的奇偶性是不变的。
  \end{Thm}
  \begin{Def}
    能被分解为偶数个对换的乘积的置换称为偶置换;能被分解为奇数个对换的乘积的置换称为奇置换。
  \end{Def}
    \begin{Thm}
    当$n \geq 2$时, $n$次奇置换的个数与$n$次偶置换的个数相等,都等于$\frac{n!}{2}$。
  \end{Thm}
      \begin{Def}
    一个集合及其在该集合上定义的若干个代数运算合成为一个代数系。
  \end{Def}
我们熟知的实数集$R$,与其上的加法运算"$+$"和乘法运算"$*$"一起构成了一个代数系,满足如下性质:
  \begin{Thm}
    设$x, y, z \in \mathbb{R}$,则
   \begin{enumerate}
   \item   $x + y = y + x$
   \item   $(x + y) + z = x + (y + z)$
   \item   $0 + x = x + 0 = x$
   \item   $(-x) + x =x + (-x) = 0$
   \item   $x * y = y * x$
   \item   $(x * y) * z = x * (y *z)$
   \item   $1 * x = x * 1 = x$
   \item   $x^{-1} * x = x * x^{-1} = 1$
   \item   $x* (y + z) = x * y + x * z$
   \item   $(y + z) * x = y * x + z * x$
    \end{enumerate}
  \end{Thm}
  \begin{Def}
    设$X$,$Y$,$Z$为任意三个非空集合。一个 从 $X\times Y$到$Z$的映射 $\phi$ 称 为 $X$与$Y$到$Z$的一个二元(代数)运算。当$X=Y=Z$时,则称$\phi$为$X$上的二元(代数)运算。
  \end{Def}
  \begin{Def}
    从集合$X$到$Y$的任一映射称为从$X$到$Y$的一元(代数)运算。如果$X=Y$,则从$X$到$X$的映射称为$X$上的一元(代数)运算。
  \end{Def}
  \begin{Def}
    设$A_1, A_2, \cdots, A_n, D$为非空集合。一个从 $A_1\times A_2\times \cdots \times A_n$到$D$的映射$\phi$称为$A_1, A_2, \cdots, A_n$到$D$的一个$n$元(代数)运算。
    如果$A_1=A_2=\cdots=A_n=D=A$,则称$\phi$为$A$上的$n$元代数运算。
  \end{Def}

  \begin{Def}
    设“$\circ$”为集合$X$上的一个二元代数运算。如果$\forall a, b \in X$,恒有\\$a \circ b = b \circ a$, 则称二元代数运算“$\circ$”满足交换律。
  \end{Def}
  \begin{Def}
    设“$\circ$”为集合$X$上的一个二元代数运算。如果$\forall a, b, c \in X$,恒有$(a \circ b) \circ c = a \circ (b \circ c)$, 则称二元代数运算“$\circ$”满足结合律。
  \end{Def}
  \begin{Def}
    设“+”与“$\circ$”为集合$X$上的两个二元代数运算。\\如果$\forall a, b, c \in X$,恒有\[a \circ (b + c) = a \circ b + a \circ c,\] 则称二元代数运算“$\circ$”对“$+$”满足左分配律。
    如果$\forall a, b, c \in X$,恒有\[(b + c)\circ a = b \circ a + c \circ a,\] 则称二元代数运算“$\circ$”对“$+$”满足右分配律。
  \end{Def}
  \begin{Def}
    设$(X, \circ)$为一个代数系。如果存在一个元素$e\in X$使得对任意的$x\in X$恒有$e\circ x = x \circ e = x$, 则称$e$为“$\circ$”的单位元素。
  \end{Def}
  \begin{Def}
    设$(X, \circ)$为一个代数系,“$\circ$”有单位元素$e$,$a\in X$,如果$\exists b\in X$使得\[a\circ b = b \circ a = e,\]  则称$b$为$a$的逆元素。
  \end{Def}
  \begin{Def}
    设$(S,+)$与$(T, \oplus)$为两个代数系。如果存在一个一一对应$\phi:S\to T$, 使得$\forall x, y \in S$,有
    \begin{align*}
      \phi(x+y) &= \phi(x) \oplus \phi(y),
    \end{align*}
    则称代数系$(S,+)$与$(T, \oplus)$同构,并记为$S\cong T$, $\phi$称为这两个代数系之间的一个同构。
  \end{Def}
  \begin{Def}
    设$(S,+, \circ)$与$(T, \oplus, *)$为两个代数系。如果存在一个一一对应$\phi:S\to T$, 使得$\forall x, y \in S$,有
    \begin{align*}
      \phi(x+y) &= \phi(x) \oplus \phi(y),\\
      \phi(x\circ y)&= \phi(x) * \phi(y),
    \end{align*}
    则称代数系$(S,+,\circ)$与$(T, \oplus, *)$同构,并记为$S\cong T$, $\phi$称为这两个代数系之间的一个同构。
  \end{Def}
  \begin{tabular}{cc|c}
    p& q& p $\land$ q\\
    \hline
    T&T&T\\
    T&F&F\\
    F&T&F\\
    F&F&F\\
  \end{tabular}\hspace{1cm}
  \begin{tabular}{cc|c}
    p& q& p $\lor$ q\\
    \hline
    T&T&T\\
    T&F&T\\
    F&T&T\\
    F&F&F\\
  \end{tabular}\hspace{1cm}
  \begin{tabular}{c|c}
    p& $\lnot$ p\\
    \hline
    T&F\\
    F&T\\
  \end{tabular}

  \vspace{1cm}
    \begin{tabular}{cc|c}
    x& y& x $\land$ y\\
    \hline
    1&1&1\\
    1&0&0\\
    0&1&0\\
    0&0&0\\
  \end{tabular}\hspace{1cm}
  \begin{tabular}{cc|c}
    x& y& x $\lor$ y\\
    \hline
    1&1&1\\
    1&0&1\\
    0&1&1\\
    0&0&0\\
  \end{tabular}\hspace{1cm}
  \begin{tabular}{c|c}
    x& $\bar{x}$\\
    \hline
    1&0\\
    0&1\\
  \end{tabular}
  
代数系$(\{T,F\},\land,\lor,\lnot)$与$(\{1,0\},\land, \lor,\bar{ })$是同构的。
  \begin{Def}
    设$X$为一个集合,$E \subseteq X$。 $E$的特征函数$\chi_E:X\to \{0,1\}$定义为
    \begin{equation*}
      \chi_E(x)=
      \begin{cases}
        1 & \text{如果} x \in E,\\
        0 & \text{如果} x \notin E.
      \end{cases}
    \end{equation*}
  \end{Def}
  \begin{Def}
    令$Ch(X) = \{\chi |\chi:X \to \{0,1\}\}$。
    $\forall \chi, \chi' \in Ch(X)$及$x \in X$,
    \begin{align}
      (\chi \lor \chi')(x) &= \chi(x) \lor \chi'(x)\nonumber\\
      (\chi \land \chi')(x) &= \chi(x) \land \chi'(x)\nonumber\\
      \bar{\chi}(x) &=   \overline{\chi(x)}
    \end{align}
  \end{Def}
  \begin{Thm}
    设$X$是一个集合,则代数系$(2^X, \cup, \cap, ^c)$与$(Ch(X), \lor, \land, \bar{} \ )$同构。
  \end{Thm}

  \begin{Thm}[鸽笼原理]
    如果把$n+1$个物体放到$n$个盒子里,则必有一个盒子里至少放了两个物体。
  \end{Thm}
  

    \begin{Exercise}
  设$X=\{a,b,c\}, Y=\{0,1\}, Z=\{2,3\}$。$f:X \to Y, f(a) = f(b) = 0, f(c) = 1$;
  $g:Y\to Z, g(0) = 2, g(1) = 3$。试求$g\circ f$。
  \end{Exercise}
  \begin{Exercise}
    设$f:X \to Y$,$C \subseteq Y$,$D \subseteq Y$,证明

    $f^{-1}(C \setminus D) = f^{-1}(C) \setminus f^{-1}(D)$
  \end{Exercise}
    \begin{Exercise}
    设$f:X \to Y$,$A \subseteq X$,$B \subseteq X$,证明

    $f(A \setminus B) \supseteq f(A) \setminus f(B)$
    
  \end{Exercise}
  \begin{Exercise}
    设$f:X\to Y$,$A \subseteq X$,则$(f(A))^c \subseteq f(A^c)$成立吗?$ f(A^c)\subseteq (f(A))^c$成立吗?
  \end{Exercise}
  \begin{Exercise}
    设$f:X\to Y$, 证明:$f$为满射当且仅当$\forall E \in 2^Y, f(f^{-1}(E)) = E$。
  \end{Exercise}

  \begin{Exercise}
    设$f:X\to Y$, 证明:$f$为单射当且仅当$\forall F \in 2^X, f^{-1}(f(F)) = F$。    
  \end{Exercise}
    \begin{Exercise}
    设$f:X \to Y$,$g:Y \to Z$,$A \subseteq Z$,证明:$(gf)^{-1}(A) = f^{-1}(g^{-1}(A))$。
  \end{Exercise}
  \begin{Exercise}
    设$N=\{1,2,\ldots\}$,试构造两个映射$f:N \to N$与$g:N\to N$,使得$fg = I_N$,
    但$gf \neq I_N$。
  \end{Exercise}
 \begin{Exercise}
    设$f:X \to Y$,

    (1)如果存在唯一的一个映射$g:Y\to X$,使得$gf = I_X$,那么$f$是否可逆呢?

    (2)如果存在唯一的一个映射$g:Y\to X$,使得$fg = I_Y$,那么$f$是否可逆呢?

  \end{Exercise}
  \begin{Exercise}
    是否存在一个从集合$X$到$X$的一一对应,使得$f=f^{-1}$,但$f \neq I_X$?
  \end{Exercise}

  \begin{Exercise}
    已知$m$个整数$a_1,a_2,\ldots,a_m$,试证:存在两个整数 $k$, $l$, \\ $0\leq k < l \leq m$,使得$a_{k+1}+a_{k+2}+\ldots+a_{l}$能被$m$整除。
  \end{Exercise}


\chapter{}

%%% Local Variables:
%%% mode: latex
%%% TeX-master: "book_chapter2"
%%% End:

\end{CJK*}
\end{document}





%%% Local Variables:
%%% mode: latex
%%% TeX-master: t
%%% End:



x